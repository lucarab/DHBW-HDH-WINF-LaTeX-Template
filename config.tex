% Dokumentart auswählen: pa1, pa2, ba
\newcommand{\dokumentart}{ba}

% Deckblatt Konfiguration
\newcommand{\projecttitle}{Lorem ipsum dolor sit amet, consetetur sadipscing elitr, sed diam nonumy eirmod tempor invidunt ut labore et dolore magna aliquyam} 

\newcommand{\degreeprogram}{Studiengang Wirtschaftsinformatik}
\newcommand{\faculty}{in der Fakultät Wirtschaft}
\newcommand{\university}{an der Duale Hochschule Baden-Württemberg}
\newcommand{\universitycity}{Standort Heidenheim}

\newcommand{\studentname}{Max Mustermann}
\newcommand{\studentaddress}{Musterstraße 1}
\newcommand{\studentcity}{99999 Musterstadt}

\newcommand{\companyname}{Beispiel GmbH}
\newcommand{\companyaddress}{Musterstraße 1}
\newcommand{\companycity}{99999 Musterstadt}

\newcommand{\semester}{2. Semester}
\newcommand{\submissiondate}{01.01.25}
\newcommand{\tutorcompany}{Max Mustermann}
\newcommand{\tutoruniversity}{Erika Müller}

% Relevant für die Bachelorarbeit
\newcommand{\degree}{Bachelor of Science}

% Ehrenwörtliche Erklärung
\newcommand{\wordcount}{4217}


% Zeichencodierung und Sprache
\usepackage[utf8]{inputenc}
\usepackage[T1]{fontenc}
\usepackage[ngerman]{babel}
\usepackage{microtype}

% Schriftart: Times oder Arial
\usepackage{fontspec}
%\setmainfont{Times New Roman}
% Wenn du Arial bevorzugst:
\setmainfont{Arial}

% Seitenlayout
\usepackage[a4paper, left=3cm, right=3cm, top=2cm, bottom=2cm]{geometry}

% Absatz und Zeilenabstand
\usepackage{setspace}
\onehalfspacing         % 1,5-zeiliger Abstand
\setlength{\parskip}{6pt}   % 6pt Abstand zwischen Absätzen
\setlength{\parindent}{0pt} % Kein Einzug
\setlength{\skip\footins}{20pt}
\usepackage[
  bottom,          
  hang,           
  stable         
]{footmisc}
\setlength{\footnotemargin}{0.5cm}
\setlength{\footnotesep}{0.4cm}

\usepackage[none]{hyphenat}
\setlength{\emergencystretch}{\maxdimen}

% Blocksatz
\usepackage{ragged2e}
\justifying

% Fußnoten: kleiner und einzeilig
\usepackage{footmisc}
\renewcommand{\footnotesize}{\fontsize{10pt}{12pt}\selectfont} % 2pt kleiner als 12pt

% Nummerierung römisch für Verzeichnisse, arabisch für Textteil
\usepackage{fancyhdr}
\usepackage{etoolbox}

% Literaturverzeichnis sichtbar im Inhaltsverzeichnis
\usepackage[nottoc]{tocbibind}

% Hyperlinks
\usepackage[hidelinks]{hyperref}

% Bilder und Tabellen
\usepackage{graphicx}
\usepackage{caption}
\usepackage{float}
\usepackage{longtable}
\usepackage{booktabs}
\usepackage{amsmath}

% Deckblatt
\usepackage{xstring}
\usepackage{tabularx}

%Glossar
\usepackage[nonumberlist,acronym,toc]{glossaries}
\makeglossaries

%Literaturverzeichnis & Zitation
\usepackage[backend=biber,style=apa,maxcitenames=2,language=german]{biblatex}
\DeclareLanguageMapping{german}{german-apa}
\usepackage{csquotes}
\DefineBibliographyStrings{german}{%
  andothers = {et al.},
}
\usepackage{setspace}
\setlength{\bibparsep}{\baselineskip}
\addbibresource{literatur.bib}
